\documentclass{article}
\usepackage{expl3}

\ExplSyntaxOn

% Create a clist of author information. Each entry key-value information for one author. 
\clist_new:N {\l_author_key_value_clist}

% Define a key-value scheme for authors, to enforce that their info in formatted correctly. 
\keys_define:nn { author }{% Define list of key-value options
  firstName     .code:n = {},    
  middleName    .code:n = {},      
  lastName      .code:n = {},    
  email         .code:n = {},
  affiliation   .code:n = {},
}

\NewDocumentCommand{\addAuthor}{m}{
  \clist_map_inline:nn { #1 } { \keys_set:nn {author} {##1} }
  \clist_put_right:Nn \l_author_key_value_clist { #1 }
}
% For convenience, define \addAuthors to match the plurality of the number of authors given.
\cs_new_eq:NN \addAuthors \addAuthor

\addAuthor{
  {
    firstName = Paul,
    lastName = Wintz,
    affiliation = UCSC, 
    email = pwintz@ucsc.edu
  },
}
\addAuthors{
  {
    firstName   = Paul,
    middleName = Kenna,
    lastName    = Wintz,
    affiliation = UCSC, 
    email       = pwintz@ucsc.edu
  },
  {
    firstName   = Carlos,
    lastName    = Blackmountain,
    affiliation = UCSC,
  },
}
\begin{document}
  \section{Key-value~pairs}

  \cs_new:Nn \l_frontloader_for_each_author:nnnn {}
  \cs_set:Nn \l_frontloader_for_each_author:nnnn {
    
    % ⋘────────── Transform each author "prop" object ──────────⋙
    
    \clist_clear:N \l_transformed_authors_clist
  
    \clist_map_inline:Nn { \l_author_key_value_clist } {
      \prop_clear:N {\l_tmpa_prop}
      % Using the clist entry "##1", create a prop that contains the key-value entries.
      \prop_put_from_keyval:Nn {\l_tmpa_prop} {##1}
      \begingroup
        % \prop_if_in:NnF \l_tmpa_prop {firstName} 
        %   {
        %     \PackageWarning{package_name}{firstName is not defined}%
        %   }
        % 
          % \cs_set:cn { firstName } { Paul }
          % \cs_new:cn { firstName } { Paul } % creates \firstName, but not expl3-style

%         % We can use a inline map over the names we want to 
%         \clist_map_inline:nn { firstName, middleName, lastName } { 
%           % Since we are three levels deep, the current clist item is ####1.
%           % \def\ifFirstName{first}
%           % if\text_titlecase_all:n {####1}
%           % \def\testttt{FirstName}
% 
%           % Store the key as a lowercase string and title case string.
%           \str_set:Nx \l_key_str {####1}
%           \str_set:Nx \l_Key_str {\text_titlecase_all:n {####1}}
%           % Debugging statements
%           if\l_Key_str \\
%           \l_key_str \\
%           \prop_item:Ne \l_tmpa_prop { \str_use:N \l_key_str } \\ 
% 
%           \expandafter\def\csname \l_key_str \endcsname {
%               % Replacing #####1 here with \l_key_str does not work because the value of \l_key_str changes before the macro is evaluated. 
%               \prop_item:Ne \l_tmpa_prop { ####1 }
%           }
%           % \expandafter\def\csname if\l_Key_str \endcsname    {\prop_item:Nn \l_tmpa_prop { \l_key_str }}
%           % \ifFirstName
% 
%           \cs_undefine:c {if\l_Key_str}
%           \cs_undefine:c {ifNot\l_Key_str}
%           \cs_undefine:c {ifElse\l_Key_str}
%           % \newcommand{\csname ifEmail \endcsname}[1]{
%           %   % If the email property is provided, insert the parameter that was passed to \ifEmail.
%           %   \prop_if_in:NnT \l_tmpa_prop {email} { ####1 }
%           % }
%           % \newcommand{\csname ifNotEmail \endcsname}[1]{
%           %   % If the email property is not provided, insert the parameter that was passed to \ifNotEmail.
%           %   \prop_if_in:NnF \l_tmpa_prop {email} { ####1 }
%           % }
%           % \newcommand{\csname ifElseEmail \endcsname}[2]{
%           %   % If the email property is provided, insert the first parameter that was passed to \ifElseEmail. Otherwise, insert the second.
%           %   \prop_if_in:NnTF \l_tmpa_prop {email} { ####1 } { ####2 }
%           % }
%           \\[2em]
%          }
         
        \str_set:Ne \l_frontloader_first_name_str   {\prop_item:Nn \l_tmpa_prop {firstName}}
        \str_set:Ne \l_frontloader_middle_names_str {\prop_item:Nn \l_tmpa_prop {middleName}}
        \str_set:Ne \l_frontloader_last_name_str    {\prop_item:Nn \l_tmpa_prop {lastName}}
        \str_set:Ne \l_frontloader_first_initial_str   {\str_head:N \l_frontloader_first_name_str  }
        \str_set:Ne \l_frontloader_middle_initials_str {\str_head:N \l_frontloader_middle_names_str}
        \str_set:Ne \l_frontloader_last_initial_str    {\str_head:N \l_frontloader_last_name_str   }
        \def\firstName    {\l_frontloader_first_name_str}
        \def\middleName  {\l_frontloader_middle_names_str} 
        \def\lastName     {\l_frontloader_last_name_str}
        \def\firstInitial  {\str_head:N \l_frontloader_first_name_str  }
        \def\middleInitials{\str_head:N \l_frontloader_middle_names_str}
        \def\lastInitial   {\str_head:N \l_frontloader_last_name_str   }
        \def\email         {\prop_item:Nn \l_tmpa_prop {email}}

        % ⋘────────── Clear all of the control sequences ──────────⋙
        \cs_undefine:N \ifFirstName
        \cs_undefine:N \ifNotFirstName
        \cs_undefine:N \ifElseFirstName
        \cs_undefine:N \ifMiddleName
        \cs_undefine:N \ifNotMiddleName
        \cs_undefine:N \ifElseMiddleName
        \cs_undefine:N \ifLastName
        \cs_undefine:N \ifNotLastName
        \cs_undefine:N \ifElseLastName
        \cs_undefine:N \ifFirstInitial
        \cs_undefine:N \ifNotFirstInitial
        \cs_undefine:N \ifElseFirstInitial
        \cs_undefine:N \ifMiddleInitial
        \cs_undefine:N \ifNotMiddleInitial
        \cs_undefine:N \ifElseMiddleInitial
        \cs_undefine:N \ifLastInitial
        \cs_undefine:N \ifNotLastInitial
        \cs_undefine:N \ifElseLastInitial
        \cs_undefine:N \ifEmail
        \cs_undefine:N \ifNotEmail
        \cs_undefine:N \ifElseEmail

        % ⋘────────── FirstName conditionals ──────────⋙
        \newcommand{\ifFirstName}[1]{
          % If the firstName property is provided, insert the parameter that was passed to \ifFirstName.
          \prop_if_in:NnT \l_tmpa_prop {firstName} { ####1 }
        }
        \newcommand{\ifNotFirstName}[1]{
          % If the firstName property is not provided, insert the parameter that was passed to \ifNotFirstName.
          \prop_if_in:NnF \l_tmpa_prop {firstName} { ####1 }
        }
        \newcommand{\ifElseFirstName}[2]{
          % If the firstName property is provided, insert the first parameter that was passed to \ifElseFirstName. Otherwise, insert the second.
          \prop_if_in:NnTF \l_tmpa_prop {firstName} { ####1 } { ####2 }
        }

        % ⋘────────── MiddleName conditionals ──────────⋙
        \newcommand{\ifMiddleName}[1]{
          % If the middleName property is provided, insert the parameter that was passed to \ifMiddleName.
          \prop_if_in:NnT \l_tmpa_prop {middleName} { ####1 }
        }
        \newcommand{\ifNotMiddleName}[1]{
          % If the middleName property is not provided, insert the parameter that was passed to \ifNotMiddleName.
          \prop_if_in:NnF \l_tmpa_prop {middleName} { ####1 }
        }
        \newcommand{\ifElseMiddleName}[2]{
          % If the middleName property is provided, insert the first parameter that was passed to \ifElseMiddleName. Otherwise, insert the second.
          \prop_if_in:NnTF \l_tmpa_prop {middleName} { ####1 } { ####2 }
        }

        % ⋘────────── LastName conditionals ──────────⋙
        \newcommand{\ifLastName}[1]{
          % If the lastName property is provided, insert the parameter that was passed to \ifLastName.
          \prop_if_in:NnT \l_tmpa_prop {lastName} { ####1 }
        }
        \newcommand{\ifNotLastName}[1]{
          % If the lastName property is not provided, insert the parameter that was passed to \ifNotLastName.
          \prop_if_in:NnF \l_tmpa_prop {lastName} { ####1 }
        }
        \newcommand{\ifElseLastName}[2]{
          % If the lastName property is provided, insert the first parameter that was passed to \ifElseLastName. Otherwise, insert the second.
          \prop_if_in:NnTF \l_tmpa_prop {lastName} { ####1 } { ####2 }
        }

        % ⋘────────── Email conditionals ──────────⋙
        \newcommand{\ifEmail}[1]{
          % If the email property is provided, insert the parameter that was passed to \ifEmail.
          \prop_if_in:NnT \l_tmpa_prop {email} { ####1 }
        }
        \newcommand{\ifNotEmail}[1]{
          % If the email property is not provided, insert the parameter that was passed to \ifNotEmail.
          \prop_if_in:NnF \l_tmpa_prop {email} { ####1 }
        }
        \newcommand{\ifElseEmail}[2]{
          % If the email property is provided, insert the first parameter that was passed to \ifElseEmail. Otherwise, insert the second.
          \prop_if_in:NnTF \l_tmpa_prop {email} { ####1 } { ####2 }
        }
        

        % Expand the "#1" argument, which stores the users template for the author, and which can use the macros \firstName, \lastName, etc. The "x" in ":Nx" indicates eXpansion.
        \tl_set:Nx \l_tmpa_tl { #1 }
        
        % Add the processed item to the list.
        \clist_gput_right:NV \l_transformed_authors_clist \l_tmpa_tl
      \endgroup
    }
    % Join the elements of the clist using the separators provided by the user (or defaults).
    \clist_use:Nnnn \l_transformed_authors_clist { #2 } { #3 }{ #4 }

%     \clist_map_inline:Nn { \l_author_key_value_clist } {
%       % Ensure that \l_tmpa_prop exists and is empty.
%       \prop_clear:N {\l_tmpa_prop}
%       \prop_put_from_keyval:Nn {\l_tmpa_prop} {##1}
% 
%       \begingroup
%         \def\firstName{\prop_item:Nn \l_tmpa_prop {firstName}}
%         \def\middleName{\prop_item:Nn \l_tmpa_prop {middleName}}
%         \def\lastName{\prop_item:Nn \l_tmpa_prop {lastName}}
%         \def\firstInitial{\prop_item:Nn \l_tmpa_prop {firstInitial}}
%         \def\middleInitials{\prop_item:Nn \l_tmpa_prop {middleInitials}}
%         \def\lastInitial{\prop_item:Nn \l_tmpa_prop {lastInitial}}
%         \def\email{\prop_item:Nn \l_tmpa_prop {email}}
%         % \firstName
%         % #1
%         #1 
%       \endgroup
%       
%     }
  }
  \NewDocumentCommand{\forEachAuthor}{ o m }{
    % Create variables to store separator options.      Defaults
    \tl_set:Nn \l_frontloader_author_separator_tl       { ~and~ }
    \tl_set:Nn \l_frontloader_author_final_separator_tl { ,~ }
    \tl_set:Nn \l_frontloader_author_two_separator_tl   { ~and~ }
    \IfNoValueF {#1} {
      \keys_define:nn { frontloader/forEachAuthorOptions }{% Define list of key-value options
        % ───── Define Token List Key-Value Options ─────
        separator             .tl_set:N  = {\l_frontloader_author_separator_tl},
        separator      .value_required:n = true, % If key is used it must have "=value".
        finalSeparator        .tl_set:N  = {\l_frontloader_author_final_separator_tl},
        finalSeparator .value_required:n = true, % If key is used it must have "=value".
        twoSeparator          .tl_set:N  = {\l_frontloader_author_two_separator_tl},
        twoSeparator   .value_required:n = true, % If key is used it must have "=value".
      }
      \keys_set:nn {frontloader/forEachAuthorOptions} { #1 }
    }
    \l_frontloader_for_each_author:nnnn{#2}
      {\l_frontloader_author_two_separator_tl}
      {\l_frontloader_author_separator_tl}
      {\l_frontloader_author_final_separator_tl}
  }
  
  \section{For~each~author}
  Count:~\clist_count:N \l_author_key_value_clist \\
  \ExplSyntaxOff
  \forEachAuthor[
    separator     = {,\\},
    finalSeparator= {, and\\},
    twoSeparator  = { and } 
  ]{
    Hello \firstName{} \middleName{} \lastName{}\ifEmail{ (\email{})}!
    Initials: \firstInitial{}\middleInitials{}\lastInitial{}
    % \ifElseEmail{Has Email}{Does not have email}
  }
  % \ifEmail{}
  
\end{document}